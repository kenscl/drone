\documentclass[10pt,a4paper]{article}
\usepackage{fullpage}
\usepackage{amsmath,amsfonts,amssymb}
\usepackage[shortlabels]{enumitem}
\usepackage{booktabs}
\usepackage{graphicx}
\usepackage{xcolor}
\usepackage{hyperref}
\usepackage{ulem}
\usepackage{tikz}
\usetikzlibrary{positioning}
\newcommand{\nl}{\vspace{12pt}}
\usepackage[utf8]{inputenc}
\usepackage{float}
%\usepackage{minted}
% Catpuccin Macchiato 

\newcommand{\M}[1]{\mathbf{#1}}
\newcommand{\V}[1]{\mathbf{#1}}
\newcommand{\Vh}[1]{\mathbf{\hat{#1}}}
\newcommand{\norm}[1]{\left | \left | #1 \right | \right |}
\newcommand{\RR}{\mathbbm{R}}        % set of real numbers

\setlength{\parindent}{0pt}

\begin{document}

\title{Extended Kalman Filter}

\maketitle

\section{Kalamn Filter}
The goal of the Kalamn Filter is to estimate the Attitude of the IMU.
The filter is defined by the following steps.
\begin{align}
    \V x_{t|t-1} &= \M F_t \V x_{t-1|t-1} \\ 
    \M P_{t|t-1} &= \M F_t \M P_{t|t-1} \M F_t^T + \M Q_{t-1}
\end{align}
and update step:
\begin{align}
    \boldsymbol \nu &=  \V z_t - \M h_t( \V x_{t|t-1} )\\
    \V K_t &= \M P_{t|t-1} \M H_t^T \left( \M H_t \M P_{t|t-1} \M H_t^T + \M R_t \right)^{-1} \\
    \V x_{t|t} &= \V x_{t|t-1} + \V K_t \boldsymbol \nu \\
    \M P_{t|t} &= \left( \M I - \V K_t \M H_t \right) \M P_{t|t-1}
\end{align}


\section{Kinematics}
\subsection {Quaternion Kinematics}
The goal is to change the Quaternion over time through the angular rate vector $\V \omega$.
Typically this is expressed as: 
\begin{align}
  \Vh{ \dot q} = \frac {\partial \Vh q} {\partial t} = \frac 1 2 \V \omega \Vh q
\end{align}
This can be expanded to: 
\begin{align}
  \frac 1 2 \V \omega \Vh q &= \frac 1 2
                              \begin{bmatrix}
                                0 - \omega_x q_i - \omega_y q_j - \omega_z q_k \\ 
                                0 + \omega_x q_w + \omega_y q_k - \omega_z q_j \\ 
                                0 - \omega_x q_k + \omega_y q_w + \omega_z q_i \\ 
                                0 + \omega_x q_j - \omega_y q_i - \omega_z q_w \\ 
                              \end{bmatrix} = \frac 1 2 \begin{bmatrix}
                                0 - \omega_x q_i - \omega_y q_j - \omega_z q_k \\ 
                                \omega_x q_w + 0 - \omega_z q_j + \omega_y q_k \\ 
                                \omega_y q_w + \omega_z q_i + 0 - \omega_x q_k \\ 
                                 - \omega_z q_w - \omega_y q_i + \omega_x q_j + 0 \\ 
                              \end{bmatrix} = \frac 1 2 \begin{bmatrix}
                                0 & - \omega_x  & - \omega_y  & - \omega_z  \\ 
                                \omega_x  &  0 & - \omega_z  &  \omega_y  \\ 
                                \omega_y  &  \omega_z  &  0 & - \omega_x  \\ 
                                 - \omega_z  & - \omega_y  &  \omega_x  &  0 \\ 
                              \end{bmatrix} \Vh q 
\end{align}

\section{Sensor Models}

\subsection{Gyroscope}
The general gyroscope model is:
\begin{align}
  \V m_g = \V m_B + \omega_g + \V v_g
\end{align}
Where $\V m_B$ is the real angular rate in the body frame, $\omega_g$ is the gyroscope bias and $\V v_g$ is the noise.
\subsection{Accelerometer}
\begin{align}
  \V m_a =  \M A (\V m_B - \V g_B) + \omega_a + \V v_a
\end{align}
Where $\V m_B$ is the real acceleration in the body frame, $\V g_B$ is the gravitational vector in the body frame, $\omega_g$ is the bias and $\V v_g$ is the noise.

\subsection{Magnetometer}
\begin{align}
  \V m_m = \V m_B + \omega_m + \V v_m
\end{align}
Where $\V m_m$ is the real magnetic filed in the body frame, $\omega_m$ is the bias and $\V v_m$ is the noise.

In actuality the Magnetometer measurement is reduced to a Yaw pseudo measurement with the aid of the Accelerometer measurements.
First the roll $\phi$ and pitch $\theta$ will be determined with the help of the corrected Magnetometer measurement $\V m_{a,c}$.
\begin{align}
  \phi &= atan2(m_{a,c,y}, - m_{a,c,z}) \\ 
  \theta &= atan\left (\frac {- m_{a,c,x}}{\sqrt{m_{a,c,y}^2 + m_{a,c,z}^2}} \right) \\ 
\end{align}
These are then used to calculate the compensated magnetic field vector $\V m_{m,c}$ (z is omitted)
\begin{align}
  m_{m,c,x} &= m_{m,x} \cdot \cos {\theta} + m_{m,z} \cdot \sin {\theta} \\
  m_{m,c,y} &= m_{m,x} \cdot \sin {\phi} \sin {\theta} + m_{m,y} \cdot \sin {\phi} - m_{m,z} \cdot \sin {\phi} \cos {\theta}\\
\end{align}
The compensated Vector is then used to calculate the yaw.
\begin{align}
  \psi = atan2(- m_{m,c,y}, m_{m,c,x})
\end{align}

\section{Measurement Prediction}

\subsection{Accelerometer}
Generally the prediction is that the gravitational vector points downwards.
Hence,
\begin{align}
  \V {h_a} = \Vh q^* \cdot \Vh g_g \cdot \Vh q = \Vh q^*
  \begin{bmatrix}
    0 \\ 0 \\ 0 \\ 1
  \end{bmatrix} \Vh q = \M R_q^{-1 }
\begin{bmatrix}
    0 \\ 0 \\ 1
  \end{bmatrix}
\end{align}
Here $\M {R_q}$ is the rotor from the current attitude.
Since the gravitational vector is mostly 0 and that $\M R^{-1} = \M R^{\mathsf {T}}$ for orthogonal matrices the computation can be simplified:
\begin{align}
  \V {h_a} = \begin{bmatrix}
    2 (q_i \cdot q_k - q_w \cdot q_j) \\
    2 (q_j \cdot q_k + q_w \cdot q_i) \\
    1 - 2 (q_i^2 + q_j^2)
  \end{bmatrix}
\end{align}

For the observation matrix the following then follows:
\begin{align}
 \M H_a =  \frac { \partial \V {h_a}}  {\partial \V x}  =
  \begin{bmatrix}
    - 2 q_j & 2 q_k & - 2 q_w & 2 q_i & 0 & 0 & 0 & 0 & 0 & 0 \\
    2 q_i & 2 q_w &  2 q_k & 2 q_j & 0 & 0 & 0 & 0 & 0 & 0 \\
    0 & - 4 q_i & - 4 q_j & 0 & 0 & 0 & 0 & 0 & 0 & 0 \\
  \end{bmatrix}
\end{align}
\subsection{Magnetometer}Since the magnetometer is only used to compute the yaw of the sensor a projection to the x-y plane is necessary to eliminate negative influence the z axis.
\begin{align}
  \V m_B = \M R^{-1} \left ((\M R \cdot \V m_B) \cdot
  \begin{bmatrix}
    1 \\ 1 \\ 0
  \end{bmatrix} \right)
\end{align}
The yaw axis is then: 
\begin{align}
  \Psi = atan2 (- m_y, m_x)
\end{align}
The prediction of $\Psi$ can be calculated from the current attitude: 
\begin{align}
  h_a = atan2 (2 (q_i \cdot q_k + q_w \cdot q_j), 1 - 2  q_j^2 - 2 q_k^2)
\end{align}
For the observation matrix the following then follows if I use my immense mathematical talents (and Wolfram Alpha):
\begin{align}
  \M H_m &=  \frac {\partial \V {h_m}}  {\partial \V x} = \begin{bmatrix}
    \frac { \partial \V {h_m}}  {\partial \V q_w} &
    \frac { \partial \V {h_m}}  {\partial \V q_i} &
    \frac { \partial \V {h_m}}  {\partial \V q_j} &
    \frac { \partial \V {h_m}}  {\partial \V q_k} &
    0 & 0 & 0& 0 & 0 & 0 \\
  \end{bmatrix} \\
    \frac { \partial \V {h_m}}  {\partial \V q_w} &= \frac { 2 q_z (-2 q_j^2 - 2 q_z^2 + 1)}{4 (q_w q_z + q_i q_j)^2 + (-2 q_j^2 - 2 q_z^2 + 1)^2} \\ 
    \frac { \partial \V {h_m}}  {\partial \V q_i} &= \frac {2 q_j (-2 q_j^2 - 2 q_z^2 + 1)}{4 (q_w q_z + q_i q_j)^2 + (-2 q_j^2 - 2 q_z^2 + 1)^2} \\
    \frac { \partial \V {h_m}}  {\partial \V q_j} &= \frac {2 (4 q_w q_j q_z + 2 q_i q_j^2 - 2 q_i q_z^2 + q_i)}{4 (q_w^2 - 1) q_z^2 + 8 q_w q_i q_j q_z + 4 q_j^2 (q_i^2 + 2 q_z^2 - 1) + 4 q_j^4 + 4 q_z^4 + 1} \\
    \frac { \partial \V {h_m}}  {\partial \V q_k} &= \frac{ q_w (-4 q_j^2 + 4 q_z^2 + 2) + 8 q_i q_j q_z}{4 (q_w^2 - 1) q_z^2 + 8 q_w q_i q_j q_z + 4 q_j^2 (q_i^2 + 2 q_z^2 - 1) + 4 q_j^4 + 4 q_z^4 + 1}
\end{align}

\section {State Model}
The EKF should predict the attitude and gyroscope bias.
Bias terms for the other sensors and non orthogonality are to be ignored.
The noise is assumed to be $\sim \mathcal{N}(0, \sigma^2)$.
Therefore: 
\begin{align}
  \V {x_k} =
  \begin{bmatrix}
    \Vh q \\ \V \omega \\ \V b
  \end{bmatrix}
\end{align}
Where $\Vh q$ is the attitude Quaternion and $\V b$ is the bias.

Notice that:
\begin{align}
  \V {q_{k+1}} &= \V{q_k} + \frac 1 2
  \begin{bmatrix}
    0 & - \omega_x  & - \omega_y  & - \omega_z  \\ 
    \omega_x  &  0 & - \omega_z  &  \omega_y   \\ 
    \omega_y  &  \omega_z  &  0 & - \omega_x \\ 
    - \omega_z  & - \omega_y  &  \omega_x  0 \\ 
  \end{bmatrix} \cdot \V {q_k} \\ 
  \V \omega &= \V m_B - \omega_g \\
  \V b &= \omega_g
\end{align}

Hence, 
\begin{align}
  \V {x_{k+1}} = \V f (k, x) = 
  \begin{bmatrix}
    \V {q_{k+1}} \\ 
    \V \omega \\
    \V b
  \end{bmatrix}
\end{align}

The state transition matrix is then the Jacobain of $\V f$:
\begin{align}
  \V {F_{k}} =  \frac { \partial \V f (k, x) } {\partial x} = 
  \begin{bmatrix}
    \frac {\partial \V {q_{k+1}}} {\partial q} & \frac {\partial \V {q_{k+1}}} {\partial \omega} & 0 \\ 
    0 & 0 & \frac {\partial \V \omega} {\partial \omega} \\
    0 & 0 & \frac {\partial \V b} {\partial \omega} \\
  \end{bmatrix}
\end{align}

\section{Noise}
The noise of the Accelerometer and the Magnetometer are trivial: 
\begin{align}
  \M R_m &= \M I_3 \sigma_m^2 \\
  \M R_\Psi &= \sigma_A^2 \\
\end{align}
The process Noise is less trivial.
The noise response of the system
\begin{align}
  \M F_R = \frac {\partial f_k} {\partial v} =
  \begin{bmatrix}
    0_{4,3} & 0_{4,3} \\
    I_3 & 0_3 \\
    0_3 & I_3
  \end{bmatrix}
\end{align}
So,
\begin{align}
  \M Q = F_R U F_R^T
\end{align}


\section{Initial Values}
Dont know yet, eh
\end{document}
