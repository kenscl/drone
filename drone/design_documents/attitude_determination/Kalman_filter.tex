\documentclass[10pt,a4paper]{article}
\usepackage{fullpage}
\usepackage{amsmath,amsfonts,amssymb}
\usepackage[shortlabels]{enumitem}
\usepackage{booktabs}
\usepackage{graphicx}
\usepackage{xcolor}
\usepackage{hyperref}
\usepackage{ulem}
\usepackage{tikz}
\usetikzlibrary{positioning}
\newcommand{\nl}{\vspace{12pt}}
\usepackage[utf8]{inputenc}
\usepackage{float}
% \usepackage{minted}
% Catpuccin Macchiato 

\newcommand{\M}[1]{\mathbf{#1}}
\newcommand{\V}[1]{\mathbf{#1}}
\newcommand{\Vh}[1]{\mathbf{\hat{#1}}}
\newcommand{\norm}[1]{\left | \left | #1 \right | \right |}
\newcommand{\RR}{\mathbbm{R}}        % set of real numbers

\setlength{\parindent}{0pt}

\begin{document}

\title{Extended Kalman Filter}

\maketitle

\section{Kalamn Filter}
We have a system defined as:
\begin{align}
  \V x_ t&= f(\V x_{t-1}, \V u_{t-1}) - \V \omega_{t-1} \\
  \V z_ t&= h(\V x_t) + \V v_t
\end{align}
The goal of the Kalamn Filter is to estimate the Attitude of the IMU.
The filter is defined by the following steps.
\begin{align}
  \V x_{t|t-1} &= f(\V x_{t-1 | t-1}, \V u_{t-1}) \\ 
  \M P_{t|t-1} &= \M F_t \M P_{t|t-1} \M F_t^T + \M Q_{t-1}
\end{align}
and update step:
\begin{align}
  \boldsymbol \nu &=  \V z_t - \M h_t( \V x_{t|t-1} )\\
  \V K_t &= \M P_{t|t-1} \M H_t^T \left( \M H_t \M P_{t|t-1} \M H_t^T + \M R_t \right)^{-1} \\
  \V x_{t|t} &= \V x_{t|t-1} + \V K_t \boldsymbol \nu \\
  \M P_{t|t} &= \left( \M I - \V K_t \M H_t \right) \M P_{t|t-1}
\end{align}


\section{Kinematics}
\subsection {Quaternion Kinematics}
The goal is to change the Quaternion over time through the angular rate vector $\V \omega$.
Typically this is expressed as: 
\begin{align}
  \Vh{ \dot q} = \frac {\partial \Vh q} {\partial t} = \frac 1 2 \V \omega \Vh q
\end{align}
This can be expanded to: 
\begin{align}
  \frac 1 2 \V \omega \Vh q &= \frac 1 2
                              \begin{bmatrix}
                                0 - \omega_x q_i - \omega_y q_j - \omega_z q_k \\ 
                                0 + \omega_x q_w + \omega_y q_k - \omega_z q_j \\ 
                                0 - \omega_x q_k + \omega_y q_w + \omega_z q_i \\ 
                                0 + \omega_x q_j - \omega_y q_i - \omega_z q_w \\ 
                              \end{bmatrix} = \frac 1 2 \begin{bmatrix}
                                0 - \omega_x q_i - \omega_y q_j - \omega_z q_k \\ 
                                \omega_x q_w + 0 - \omega_z q_j + \omega_y q_k \\ 
                                \omega_y q_w + \omega_z q_i + 0 - \omega_x q_k \\ 
                                - \omega_z q_w - \omega_y q_i + \omega_x q_j + 0 \\ 
                              \end{bmatrix} = \frac 1 2 \begin{bmatrix}
                                0 & - \omega_x  & - \omega_y  & - \omega_z  \\ 
                                \omega_x  &  0 & - \omega_z  &  \omega_y  \\ 
                                \omega_y  &  \omega_z  &  0 & - \omega_x  \\ 
                                - \omega_z  & - \omega_y  &  \omega_x  &  0 \\ 
                              \end{bmatrix} \Vh q 
\end{align}

\section{Sensor Models}

\subsection{Gyroscope}
The general gyroscope model is:
\begin{align}
  \V m_g = \V m_B + \omega_g + \V v_g
\end{align}
Where $\V m_B$ is the real angular rate in the body frame, $\omega_g$ is the gyroscope bias and $\V v_g$ is the noise.
\subsection{Accelerometer}
\begin{align}
  \V m_a =  \M A (\V m_B - \V g_B) + \omega_a + \V v_a
\end{align}
Where $\V m_B$ is the real acceleration in the body frame, $\V g_B$ is the gravitational vector in the body frame, $\omega_g$ is the bias and $\V v_g$ is the noise.

\subsection{Magnetometer}
\begin{align}
  \V m_m = \V m_B + \omega_m + \V v_m
\end{align}
Where $\V m_m$ is the real magnetic filed in the body frame, $\omega_m$ is the bias and $\V v_m$ is the noise.

In actuality the Magnetometer measurement is reduced to a Yaw pseudo measurement with the aid of the Accelerometer measurements.
First the roll $\phi$ and pitch $\theta$ will be determined with the help of the corrected Magnetometer measurement $\V m_{a,c}$.
\begin{align}
  \phi &= atan2(m_{a,c,y}, m_{a,c,z}) \\ 
  \theta &= atan\left (\frac {- m_{a,c,x}}{\sqrt{m_{a,c,y}^2 + m_{a,c,z}^2}} \right) \\ 
\end{align}
These are then used to calculate the compensated magnetic field vector $\V m_{m,c}$ (z is omitted)
\begin{align}
  m_{m,c,x} &= m_{m,x} \cdot \cos {\theta} + m_{m,z} \cdot \sin {\theta} \\
  m_{m,c,y} &= m_{m,x} \cdot \sin {\phi} \sin {\theta} + m_{m,y} \cdot \cos {\phi} - m_{m,z} \cdot \sin {\phi} \cos {\theta}\\
\end{align}
The compensated Vector is then used to calculate the yaw.
\begin{align}
  \psi = atan2(- m_{m,c,y}, m_{m,c,x})
\end{align}

\section{Measurement Prediction}

\subsection{Accelerometer}
Generally the prediction is that the gravitational vector points downwards.
Hence,
\begin{align}
  \V {h_a} = \Vh q^* \cdot \Vh g_g \cdot \Vh q = \Vh q^*
  \begin{bmatrix}
    0 \\ 0 \\ 0 \\ 1
  \end{bmatrix} \Vh q = \M R_q^{-1 }
  \begin{bmatrix}
    0 \\ 0 \\ 1
  \end{bmatrix}
\end{align}
Here $\M {R_q}$ is the rotor from the current attitude.
Since the gravitational vector is mostly 0 and that $\M R^{-1} = \M R^{\mathsf {T}}$ for orthogonal matrices the computation can be simplified:
\begin{align}
  \V {h_a} = \begin{bmatrix}
    2 (q_i \cdot q_k - q_w \cdot q_j) \\
    2 (q_j \cdot q_k + q_w \cdot q_i) \\
    q_w^2 - q_i^2 - q_j^2 + q_k^2
  \end{bmatrix}
\end{align}

For the observation matrix the following then follows:
\begin{align}
  \M H_a =  \frac { \partial \V {h_a}}  {\partial \V x}  =
  \begin{bmatrix}
    - 2 q_j & 2 q_k & - 2 q_w & 2 q_i & 0 & 0 & 0 \\
    2 q_i & 2 q_w &  2 q_k & 2 q_j & 0 & 0 & 0 \\
    2 q_w & - 2 q_i & - 2 q_j & 2 q_k & 0 & 0 & 0 \\
  \end{bmatrix}
\end{align}
\subsection{Magnetometer}Since the magnetometer is only used to compute the yaw of the sensor a projection to the x-y plane is necessary to eliminate negative influence the z axis.
\begin{align}
  \V m_B = \M R^{-1} \left ((\M R \cdot \V m_B) \cdot
  \begin{bmatrix}
    1 \\ 1 \\ 0
  \end{bmatrix} \right)
\end{align}
The yaw axis is then: 
\begin{align}
  \Psi = atan2 (- m_y, m_x)
\end{align}
The prediction of $\Psi$ can be calculated from the current attitude: 
\begin{align}
  h_a = atan2 (2 (q_w \cdot q_k + q_i \cdot q_j), 1 - 2  q_j^2 - 2 q_k^2)
\end{align}
We can rewrite this as:
\begin{align}
  h_a &= atan2 (A, B) \\
  A &= 2 (q_w \cdot q_k + q_i \cdot q_j) \\
  B &= 1 - 2  q_j^2 - 2 q_k^2 \\
\end{align}
Then the standard derivitive is:
\begin{align}
  \frac {\partial h_a} {\partial q_n} = \frac {B} {A^2 + B^2} \frac {\partial A} {\partial x} - \frac {A} {A^2 + B^2} \frac {\partial B} {\partial x}
\end{align}
So in total: 
\begin{align}
  \M H_m &=  \frac {\partial \V {h_m}}  {\partial \V x} = \begin{bmatrix}
    \frac { \partial \V {h_m}}  {\partial \V q_w} &
                                                    \frac { \partial \V {h_m}}  {\partial \V q_i} &
                                                                                                    \frac { \partial \V {h_m}}  {\partial \V q_j} &
                                                                                                                                                    \frac { \partial \V {h_m}}  {\partial \V q_k} &
                                                                                                                                                                                                    0 & 0 & 0 \\
  \end{bmatrix} \\
  \frac { \partial \V {h_m}}  {\partial \V q_w} &= \frac B {A^2 + B^2} 2 q_k \\ 
  \frac { \partial \V {h_m}}  {\partial \V q_i} &= \frac B {A^2 + B^2} 2 q_j\\
  \frac { \partial \V {h_m}}  {\partial \V q_j} &= \frac {2Bq_i + 4Aq_j} {A^2 + B^2}   \\
  \frac { \partial \V {h_m}}  {\partial \V q_k} &= \frac {2Bq_w + 4q_k} {A^2 + B^2}
\end{align}

\section {State Model}
The EKF should predict the attitude and gyroscope bias.
Bias terms for the other sensors and non orthogonality are to be ignored.
The noise is assumed to be $\sim \mathcal{N}(0, \sigma^2)$.
Therefore: 
\begin{align}
  \V {x_k} =
  \begin{bmatrix}
    \Vh q \\ \V b
  \end{bmatrix}
\end{align}
Where $\Vh q$ is the attitude Quaternion and $\V b$ is the bias.

Notice that:
\begin{align}
  \V {q_{k+1}} &= \V{q_k} + \frac 1 2
                 \begin{bmatrix}
                   0 & - \omega_x  & - \omega_y  & - \omega_z  \\ 
                   \omega_x  &  0 & - \omega_z  &  \omega_y   \\ 
                   \omega_y  &  \omega_z  &  0 & - \omega_x \\ 
                   - \omega_z  & - \omega_y  &  \omega_x  0 \\ 
                 \end{bmatrix} \cdot \V {q_k} \Delta t \\ 
  \V b_{k + 1} &= \V b_k
\end{align}
Here $\V \omega = \V m_g - \V b_k$

Hence, 
\begin{align}
  \V {x_{k+1}} = \V f (k, x) = 
  \begin{bmatrix}
    \V {q_{k+1}} \\ 
    \V b
  \end{bmatrix}
\end{align}

The state transition matrix is then the Jacobain of $\V f$:
\begin{align}
  \V {F_{k}} =  \frac { \partial \V f (k, x) } {\partial x} = 
  \begin{bmatrix}
    \frac {\partial \V {q_{k+1}}} {\partial q} & \frac {\partial \V {q_{k+1}}} {\partial b} \\ 
    0 &  I \\
  \end{bmatrix}
\end{align}
Where: 
\begin{align}
  \frac {\partial \V {q_{k+1}}} {\partial q} &= 
                                               \V {q_{k+1}} = I + \frac 1 2
                                               \begin{bmatrix}
                                                 0 & - \omega_x  & - \omega_y  & - \omega_z  \\ 
                                                 \omega_x  &  0 & - \omega_z  &  \omega_y   \\ 
                                                 \omega_y  &  \omega_z  &  0 & - \omega_x \\ 
                                                 - \omega_z  & - \omega_y  &  \omega_x  0 \\ 
                                               \end{bmatrix}  \Delta t \\ 
  \frac {\partial \V {q_{k+1}}} {\partial b} &=
                                               \begin{bmatrix}
                                                 \frac 1 2 \begin{bmatrix}
                                                   0 & -1 & 0 & 0 \\
                                                   1 & 0 & 0 & 0 \\
                                                   0 & 0 & 0 & -1 \\
                                                   0 & 0 & 1 & 0
                                                 \end{bmatrix} q_k \Delta t & \frac 1 2 \begin{bmatrix}
                                                   0 & 0 & -1 & 0 \\
                                                   0 & 0 & 0 & 1 \\
                                                   1 & 0 & 0 & 0 \\
                                                   0 & -1 & 0 & 0
                                                 \end{bmatrix} q_k \Delta t & \frac 1 2 \begin{bmatrix}
                                                   0 & 0 & 0 & -1 \\
                                                   0 & 0 & -1 & 0 \\
                                                   0 & 1 & 0 & 0 \\
                                                   -1 & 0 & 0 & 0
                                                 \end{bmatrix} q_k \Delta t
                                               \end{bmatrix} \\
                                             &=
                                               \begin{bmatrix}
                                                 \frac 1 2 \Delta t
                                                 \begin{bmatrix}
                                                   - q_i \\
                                                   q_w \\
                                                   -q_k \\
                                                   q_j
                                                 \end{bmatrix} & 
                                                                 \frac 1 2 \Delta t
                                                                 \begin{bmatrix}
                                                                   - q_j \\
                                                                   q_k \\
                                                                   q_w \\
                                                                   - q_i
                                                                 \end{bmatrix} &
                                                                                 \frac 1 2 \Delta t
                                                                                 \begin{bmatrix}
                                                                                   - q_k \\
                                                                                   - q_j \\
                                                                                   q_i \\
                                                                                   - q_w
                                                                                 \end{bmatrix} 
                                               \end{bmatrix}
\end{align}

\section{Noise}

In an Extended Kalman Filter (EKF), two primary sources of noise are modeled:
\begin{itemize}
  \item \textbf{Process noise}, representing uncertainty in the system dynamics.
  \item \textbf{Measurement noise}, representing uncertainty in sensor observations.
\end{itemize}

\subsection{Measurement Noise}

The measurement noise arises from the accelerometer and magnetometer.  
Assuming white, zero-mean, isotropic noise, the corresponding covariance matrices are
\begin{align}
  \mathbf{R}_A &= \mathbf{I}_3 \sigma_A^2, \\
  \mathbf{R}_M &= \mathbf{I}_3 \sigma_M^2,
\end{align}
where $\sigma_A$ and $\sigma_M$ are the standard deviations of the accelerometer and magnetometer noise, respectively.

The combined measurement noise covariance used in the EKF update step is
\begin{align}
  \mathbf{R} =
  \begin{bmatrix}
    \mathbf{R}_M & \mathbf{0}_{3\times3} \\
    \mathbf{0}_{3\times3} & \mathbf{R}_A
  \end{bmatrix}
  =
  \begin{bmatrix}
    \mathbf{I}_3 \sigma_M^2 & \mathbf{0} \\
    \mathbf{0} & \mathbf{I}_3 \sigma_A^2
  \end{bmatrix}.
\end{align}

\subsection{Process Noise}

The system state includes both the quaternion and the gyroscope bias:
\begin{align}
  \mathbf{x} =
  \begin{bmatrix}
    \mathbf{q} \\[1mm]
    \mathbf{b}_g
  \end{bmatrix} \in \mathbb{R}^7,
\end{align}

The quaternion propagation is modeled as
\begin{align}
  \dot{\mathbf{q}} = \frac{1}{2} \, \boldsymbol{\Omega}(\mathbf{q}) \, (\mathbf{m}_g - \mathbf{b}_g - \mathbf{v}_g),
\end{align}
where $\mathbf{v}_g$ is the gyroscope noise and
\begin{align}
  \boldsymbol{\Omega}(\mathbf{q}) =
  \begin{bmatrix}
   -q_1 & -q_2 & -q_3 \\
    q_0 & -q_3 &  q_2 \\
    q_3 &  q_0 & -q_1 \\
   -q_2 &  q_1 &  q_0
  \end{bmatrix}.
\end{align}

The gyroscope bias is modeled as a random walk:
\begin{align}
  \dot{\mathbf{b}}_g = \mathbf{w}_b,
\end{align}
where $\mathbf{w}_b$ is zero-mean Gaussian noise.

\subsection{Discrete Process Noise Covariance}

After discretization over a time step $\Delta t$, the Jacobian of the quaternion with respect to gyro noise is
\begin{align}
  \frac{\partial \mathbf{q}_{k+1}}{\partial \mathbf{v}_g} = -\frac{1}{2} \, \boldsymbol{\Omega}(\mathbf{q}_k) \, \Delta t.
\end{align}

Assuming the gyroscope and bias noise covariances are
\begin{align}
  \mathbf{Q}_g &= \mathbf{I}_3 \sigma_g^2, &
  \mathbf{Q}_b &= \mathbf{I}_3 \sigma_b^2 \Delta t^2,
\end{align}
Then: 
\begin{align}
  \mathbf{Q} =
  \begin{bmatrix}
    \mathbf{Q}_q & \mathbf{0}_{4\times3} \\[1mm]
    \mathbf{0}_{3\times4} & \mathbf{Q}_b
  \end{bmatrix}, \quad
  \mathbf{Q}_q = \frac{1}{4} \boldsymbol{\Omega}(\mathbf{q}_k) \, \mathbf{Q}_g \, \boldsymbol{\Omega}(\mathbf{q}_k)^T \, \Delta t^2.
\end{align}



\section{Initial Values}
We generate Initial values for the roll, pitch and yaw using a method similar to the complementary filter.
For the variances we use Welford's online algorithm:
\begin{align}
  \bar x_n &= \bar x_{n-1} + \frac {x_n - \bar x_{n-1}} n \\
  M_n &= M_{n-1} + (x_n - \bar x_{n-1}) (x_n - \bar x_n) \\
  \sigma_n^2 &= \frac {M_n} {n-1}
\end{align}
\end{document}

%%% Local Variables:
%%% mode: LaTeX
%%% TeX-master: t
%%% End:
